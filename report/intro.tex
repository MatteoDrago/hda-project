% !TEX root = template.tex

\section{Introduction}
\label{sec:introduction}

During the past decade, time series classification has captured growing interest thanks to the introduction of deep learning mechanisms, such as neural networks. These tools are indeed capable of identifying and learning signal features, which are then exploited for classification, without the need of human domain-knowledge: this is a huge step forward considering that features were traditionally hand-crafted.
Human Activity Recognition (HAR) in particular has been fostered by the spread of powerful, efficient and affordable sensors, which nowadays are commonly found in mobile phones and wearable devices, with multiple applications, ranging from health care to gaming and virtual reality \cite{HAR-2013}.
Wearable sensors allow us to collect and process a huge amount of signals, which are essential for deep neural networks (DNN) to work properly: in fact, in order for them to learn and for being accurate enough to be preferred over standard machine learning approaches, we need the input training set to be heterogeneous, meaningful and representative of the problem.
For these reasons, HAR is not an easy classification problem: when dealing with on-body sensors, system performances heavily depends on human behaviour, which is a source of high variability; moreover, data collected from sensors is typically high-dimensional, multi-modal and subjected to noise, making the problem even more difficult from a machine learning perspective. 

In recent years, several ways of performing activity detection and classification have been proposed: in the literature there's no shortage of models.
The trend has been to expand the power of networks, adding more and more layers: this resulted in more accurate models, that had to face though an increasing computational complexity. The computational power however is limited in real applications and excessive complexity might translate into impracticability. Moreover, as pointed out in \cite{Chavarriaga2013} and \cite{HAR-COMP2018}, despite the proliferation of models to perform activity detection and classification, the lack of common benchmarking data, and of detailed descriptions of the proposed models, prevented a fair comparison between different solutions.

Considering than that the activity recognition problem has been already widely addressed by many authors, we decided to present a systematic comparison between two different commonly proposed types of pipeline.
The two differ on how inactivity is handled: the first one tries to learn a representation of the signals where no action is performed, adding a null class to the other activities; the second one instead splits the classification into two tasks, first deploying an activity detector that filters out inactivity signals and then classifying the remaining activities.
Our study is meant to provide a baseline for future work, giving an idea on which system could be more appealing.
In order to assess the efficiency of our models, that have been designed against the trend trying to minimize the number of trainable parameters, we used the OPPORTUNITY dataset \cite{Chavarriaga2013, ComplexAct-2010} which will be described in details in the following sections.

In conclusion, the contributions of this paper are:
\begin{itemize}
	\item overview of the latest progresses of the state of the art;
	\item implementation of those solutions;
	\item comparison of two different approaches. 
\end{itemize}
 
The paper is organized as follows: section \ref{sec:related_work} provides a summary of the latest and more important works related to our studies; in section \ref{sec:processing_architecture} we start delving into the details of how we organized our HAR architecture, step by step; section \ref{sec:model} is dedicated to the description of the dataset and to the decisions we made in the preprocessing phase; finally in section \ref{sec:learning_framework} we are ready to describe meticulously the learning framework, while sections \ref{sec:results} and \ref{sec:conclusions} are for discussion of results and for drawing our conclusions. 