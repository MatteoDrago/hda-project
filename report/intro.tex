% !TEX root = template.tex

\section{Introduction}
\label{sec:introduction}

\red{Maximum length for the whole report is 9 pages. Abstract, introduction and related works should take max two pages.}\\

During the past decade, time series classification has captured growing interest thanks to the introduction of deep learning techniques, such as neural networks. These tools indeed are capable of identify and learn signal features, which are then exploited for classification, without the need of human domain-knowledge: this is a huge step forward considering that features were traditionally hand-crafted.[]
Human Activity Recognition (HAR) in particular has been fostered by the spread of powerful, efficient and affordable sensors, which nowadays are commonly found in mobile phones and wearable devices, with multiple applications, ranging from health care to gaming and virtual reality. \cite{HAR-2013}
Wearable sensors allow us to collect and process a huge amount of signals, which are essential for deep neural networks (DNN) to work properly: in fact, in order for them to learn and being accurate enough to be preferred over standard machine learning approaches, we need the input training set to be heterogeneous, meaningful and representative of the problem.
For this reason, HAR is not an easy classification problem: when dealing with on-body sensors, system performances heavily depends on human behaviour which is a source of high variability; moreover, data collected from sensors is typically high-dimensional, multi-modal and subjected to noise, making the problem even more difficult from a machine learning perspective.
In the recent years, several models to perform activity detection and classification have been proposed [], but as pointed out in \cite{Chavarriaga2013} and [], the lack of a baseline evaluation and of structured and fixed implementation details prevented a fair comparison between different solutions.

\MR{A good way of structuring the introduction is as follows: 
\begin{itemize}
\item one paragraph to introduce your work, describing the scenario {\it at large}, its relevance, to prepare the reader to what follows and convince her/him that the paper focuses on an important setup / problem. 
\item a second paragraph where you immediately delve into the specific problem that is still to be faced, starting to point the finger towards your contribution. Here, you describe the importance of such problem, providing examples (through references) of previous solutions attempts, and of why these failed {\it to provide a complete answer}. This second paragraph should not be too long, as otherwise the reader will get bored and will abandon your paper... It should be concisely written, something like 4 to 5 lines.
\item a third paragraph were you state what you do in the paper, this should also be concisely written and to the point. A good rule of thumb is to make it max 10 lines. Here, you should state up front 1) the problem you solve, 2) its importance, 3) the technique you use, 4) stress the novelty of such technique / what you do. 5) comment on how your work / results can be reused / exploited to achieve further technical or practical goals.
\item after this, you provide an itemized list to summarize the paper contributions: maximum six items, maximum four lines each.
\item you finish up by reporting the paper structure, this should be three to four lines. It is customary to do so, although I admit it may be of little use.\\
\end{itemize}}

\MR{Lately, I tend to write introduction plus abstract within a single page. This forces me to focus on the important messages that I want to deliver about the paper, leaving out all the blah blah. \textbf{Remember:} 1) {\it less is more}, 2) writing a compact ({\it snappy}) piece of technical text is much more difficult than writing with no space constraints.}