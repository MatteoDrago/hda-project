% !TEX root = template.tex

\section{Introduction}
\label{sec:introduction}

Human Activity Recognition (HAR) has gained more and more momentum thanks to the advent of efficient and low-cost wearable sensors or mobile devices, which allow to collect and process a huge amount of signals.
Detecting and classifying human activity in fact isn't an easy task: when dealing with on-body sensors, human behaviour can make the difference in correctly or erroneously classifying a signal because it is subject to high variability; moreover, data is typically high-dimensional, multi-modal and subjected to noise, making the problem even more difficult from a machine learning perspective.
While in the past it was common to use models with hand-crafted features, increasing then the complexity of the task by requiring more domain knowledge, nowadays we can rely on deep learning techniques, which are able to automatically extract the features to be considered going beyond human understanding.
Applications are multiple, ranging from health care to gaming.

In this paper we present our system to recognize human activity, tested on a benchmark dataset, the OPPORTUNITY Human Activity dataset \cite{bibid}. Our model ...


\red{Maximum length for the whole report is 9 pages. Abstract, introduction and related works should take max two pages.}\\

\MR{A good way of structuring the introduction is as follows: 
\begin{itemize}
\item one paragraph to introduce your work, describing the scenario {\it at large}, its relevance, to prepare the reader to what follows and convince her/him that the paper focuses on an important setup / problem. 
\item a second paragraph where you immediately delve into the specific problem that is still to be faced, starting to point the finger towards your contribution. Here, you describe the importance of such problem, providing examples (through references) of previous solutions attempts, and of why these failed {\it to provide a complete answer}. This second paragraph should not be too long, as otherwise the reader will get bored and will abandon your paper... It should be concisely written, something like 4 to 5 lines.
\item a third paragraph were you state what you do in the paper, this should also be concisely written and to the point. A good rule of thumb is to make it max 10 lines. Here, you should state up front 1) the problem you solve, 2) its importance, 3) the technique you use, 4) stress the novelty of such technique / what you do. 5) comment on how your work / results can be reused / exploited to achieve further technical or practical goals.
\item after this, you provide an itemized list to summarize the paper contributions: maximum six items, maximum four lines each.
\item you finish up by reporting the paper structure, this should be three to four lines. It is customary to do so, although I admit it may be of little use.\\
\end{itemize}}

\MR{Lately, I tend to write introduction plus abstract within a single page. This forces me to focus on the important messages that I want to deliver about the paper, leaving out all the blah blah. \textbf{Remember:} 1) {\it less is more}, 2) writing a compact ({\it snappy}) piece of technical text is much more difficult than writing with no space constraints.}