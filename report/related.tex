% !TEX root = template.tex

\section{Related Work}
\label{sec:related_work}

The OPPORTUNITY activity recognition dataset is a benchmarking dataset, introduced in \cite{Roggen2010}, to overcome the lack of an evaluation setup to compare different classification systems and to provide a more exhaustive dataset compared to the others, which "are not sufficiently rich to investigate opportunistic activity recognition, where a high number of sensors is required on the body, in objects and in the environment, with a high number of activity instances". As pointed out in \cite{Chavarriaga2015} in fact, previously, several datasets were related to the activities which were to be classified: this is due to researchers acquiring signals only from sensors located in specific locations, according to the task to be performed.
The OPPORTUNITY activity recognition dataset instead has the purpose of collecting signals from an entire environment to enable a fair comparison of different learning models.

\MR{The goal of this section is to describe what has been done so far in {\it the} literature. You should focus on and briefly describe the work done in the best papers that you have read. For each you should comment on the paper's contribution, on the good and important findings of such paper and also, 1) on why these findings are not enough and 2) how these findings are improved upon / extended by the work that you do here. At the end of the section, you recap the main paper contributions (one or two, the most important ones) and how these extend / improve upon previous work. If possible, I would make this section no longer than one page, this leads to an overall {\it two pages} including abstract, introduction and related work. I believe this is a fair amount of space in most cases.}\\
\begin{itemize}
\item \MR{\textbf{References:} please follow this {\it religiously}. It will help you a lot. Use {\it bibtex} as the tool to manage the bibliography. A bibtex example file, maned {\tt biblio.bib} is also provided with this package.}

\item \MR{When referring to \textbf{conference / workshop papers}, I recommend to always include the following information: 1) author names, 2) paper title, 3) conference / workshop name, 4) conference / workshop address, 5) month, 6) year. Examples of this are: \cite{Zargham-2011}\cite{Sadler-2006}.}

\item \MR{When referring to \textbf{journal papers}, include the following information: 1) author names, 2) paper title, 3) full journal name, 4) volume, 5) number, 6) month, 7) pages, 8) year. Examples of this are: \cite{Shannon-1948}\cite{Boyd-2011}\cite{Zordan-2014}.}

\item \MR{For \textbf{books}, include the following information: 1) author names, 2) book title, 3) editor and edition, 4) year.}
\end{itemize}
%
\MR{Note that some of the above fields may not be shown when you compile the Latex file, but this depends on the bibliography settings (dictated by the specific Latex style that you load at the beginning of the document). You may decide to include additional pieces of information in a given bibliographic entry, but please, be consistent across all the entries, i.e., use the same fields. Exceptions are in the (rare) cases where some of the fields do not exist (e.g., the paper {\it number} or the {\it pages}).}