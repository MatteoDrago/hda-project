% !TEX root = template.tex

\section{Related Work}
\label{sec:related_work}
The \textbf{OPPORTUNITY} activity recognition dataset has been introduced in \cite{ComplexAct-2010} to overcome the lack of an evaluation setup, to compare different classification systems and to provide a more exhaustive dataset compared to the others, which "are not sufficiently rich to investigate opportunistic activity recognition, where a high number of sensors is required on the body, in objects and in the environment, with a high number of activity instances". As pointed out in \cite{Chavarriaga2013} in fact, previously, several datasets were related to the activities which were to be classified: this is due to researchers acquiring signals only from sensors located in specific locations, according to the task of interest.
To overcome this drawback, the \textbf{OPPORTUNITY} dataset has been gathered from a monitored, sensor rich environment \MD{aggiungere img dell'ambiente?} : objects from the scene were connected to acquisition sensors, while people participating to the session were equipped with on-body sensors\MD{; signals collected from different sensors will be described in section \ref{sec:model}}. This particular dataset has been fundamental over the past years, it provided indeed an heterogeneous and complete set of time series, perfectly suitable for different studies in the \textbf{HAR} domain. In \cite{Chavarriaga2013} they present it as a \textit{benchmark dataset}; as a demonstration, they provide the results obtained with four classification techniques (\textit{k-nearest neighbours, nearest centroid, linear discriminant analysis, quadratic discriminant analysis}) and they compare them with other works that used the same dataset. \MD{inseriamo anche i valori che ottengono nel paper per confronto?}

The authors in \cite{cao2012integrated} proposed an exhaustive framework which, besides the standard preprocessing on the activity data sequence (filling of the gaps via interpolation and data normalization), presents also a solution for the well-known class imbalance problem \cite{japkowicz2002class}. Moreover, they also include a post-processing procedure after classification consisting of a smoothing operation along the temporal axis \MD{(i dati non vengono finestrati e quindi loro li filtrano)} and of a strategic fusion procedure to integrate prediction sequences from different classifiers, in order to reduce the risk of making an erroneous classification. The classifiers used in this work consisted in a 1-layer neural network (1NN) and a Support Vector Machine (SVM, complete overview of this tool in \cite{hearst1998support}). Even for this work the \textbf{OPPORTUNITY} dataset has been used for assessing performances.

As we outlined before, the recent explosion of 