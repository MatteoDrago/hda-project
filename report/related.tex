% !TEX root = template.tex

\section{Related Work}
\label{sec:related_work}
The OPPORTUNITY activity recognition dataset is a benchmarking dataset, introduced in \cite{Roggen2010}, to overcome the lack of an evaluation setup to compare different classification systems and to provide a more exhaustive dataset compared to the others, which "are not sufficiently rich to investigate opportunistic activity recognition, where a high number of sensors is required on the body, in objects and in the environment, with a high number of activity instances". As pointed out in \cite{Chavarriaga2015} in fact, previously, several datasets were related to the activities which were to be classified: this is due to researchers acquiring signals only from sensors located in specific locations, according to the task to be performed.
The OPPORTUNITY activity recognition dataset instead has the purpose of collecting signals from an entire environment to enable a fair comparison of different learning models.