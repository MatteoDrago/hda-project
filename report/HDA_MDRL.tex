\documentclass[10pt, conference, journal]{IEEEtran}

\usepackage{algorithm}
\usepackage{algorithmicx}
\usepackage{algpseudocode}
\usepackage{amsfonts}
\usepackage{amsmath}
\usepackage{amssymb}
\usepackage[T1]{fontenc}
\usepackage[utf8]{inputenc} % need UTF-8 encoding if you're under windows or you use TeX Studio
\usepackage{xcolor}
\usepackage{mathtools}
\usepackage{graphicx}
%\usepackage{caption}
\usepackage{subcaption}
\usepackage{import}
\usepackage{multirow}
\usepackage{cite}
\usepackage[export]{adjustbox}
\usepackage{breqn}
\usepackage{mathrsfs}
\usepackage{acronym}
%\usepackage[keeplastbox]{flushend}
%\usepackage{setspace}
\usepackage{stackengine}

\renewcommand{\thetable}{\arabic{table}}
\renewcommand{\thesubtable}{\alph{subtable}}

\DeclareMathOperator*{\argmin}{arg\,min}
\DeclareMathOperator*{\argmax}{arg\,max}

\def\delequal{\mathrel{\ensurestackMath{\stackon[1pt]{=}{\scriptscriptstyle\Delta}}}}

\graphicspath{{./figures/}}
\setlength{\belowcaptionskip}{0mm}
\setlength{\textfloatsep}{8pt}

\newcommand{\eq}[1]{Eq.~\eqref{#1}}
\newcommand{\fig}[1]{Fig.~\ref{#1}}
\newcommand{\tab}[1]{Tab.~\ref{#1}}
\newcommand{\secref}[1]{Section~\ref{#1}}

\newcommand{\RomanNumeralCaps}[1]
{\MakeUppercase{\romannumeral #1}}

\newcommand\MD[1]{\textcolor{blue}{#1}}
\newcommand\RL[1]{\textcolor{red}{#1}}

%\renewcommand{\baselinestretch}{0.98}
% \renewcommand{\bottomfraction}{0.8}
% \setlength{\abovecaptionskip}{0pt}
\setlength{\columnsep}{0.2in}

% \IEEEoverridecommandlockouts\IEEEpubid{\makebox[\columnwidth]{PUT COPYRIGHT NOTICE HERE \hfill} \hspace{\columnsep}\makebox[\columnwidth]{ }} 

\title{Deep Learning Techniques for Gesture Recognition: Where to Split the Complexity}

\author{Matteo Drago, Riccardo Lincetto$^\dag$
\thanks{$^\dag$Department of Information Engineering, email: \{matteo.drago,riccardo.lincetto\}@studenti.unipd.it}
%\thanks{$^\ddag$Department of Information Engineering, email: \{riccardo .lincetto\}@studenti.unipd.it}
%\thanks{Special thanks / acknowledgement go here.}
} 

\IEEEoverridecommandlockouts

\begin{document}

\maketitle

\begin{abstract}
	
With the increasing interest in deep learning techniques and its applications, also Human Activity Recognition (HAR) saw significant improvements; before neural networks were put into practice, most of the research activities on the field relied on hand-crafted features which, however, couldn't represent nor distinguish well enough complex and articulated movements. Moreover, the use of smart devices and wearable sensors brought the challenge to another level: dealing with high-dimensional and noisy time series while assuring optimal performances requires a detailed study and, most of all, a considerable computational effort.

In our paper we present the design of an HAR architecture which implements convolutional layers in order to extract significant features from windows of samples, along with Long-Short Term Memory (LSTM) layers, suitable to exploit time dependencies among consecutive samples. For our study, we tried to minimize the collection of layers per network and thus the amount of parameters to train, which could be of great advantage in real time applications. In addition, we also decided to study how performances change if we split the process into two distinct phases: the first one that performs \textit{activity detection} while the last one \textit{activity classification}. The dataset that we used to assess the efficiency of our architectures is the OPPORTUNITY dataset.

\end{abstract}

\IEEEkeywords
Human Activity Recognition, Machine Learning, Neural Networks, Motion Detection. 
\endIEEEkeywords


\input{Intro}

% !TEX root = template.tex

\section{Related Work}
\label{sec:related_work}

\MR{The goal of this section is to describe what has been done so far in {\it the} literature. You should focus on and briefly describe the work done in the best papers that you have read. For each you should comment on the paper's contribution, on the good and important findings of such paper and also, 1) on why these findings are not enough and 2) how these findings are improved upon / extended by the work that you do here. At the end of the section, you recap the main paper contributions (one or two, the most important ones) and how these extend / improve upon previous work. If possible, I would make this section no longer than one page, this leads to an overall {\it two pages} including abstract, introduction and related work. I believe this is a fair amount of space in most cases.}\\
\begin{itemize}
\item \MR{\textbf{References:} please follow this {\it religiously}. It will help you a lot. Use {\it bibtex} as the tool to manage the bibliography. A bibtex example file, maned {\tt biblio.bib} is also provided with this package.}

\item \MR{When referring to \textbf{conference / workshop papers}, I recommend to always include the following information: 1) author names, 2) paper title, 3) conference / workshop name, 4) conference / workshop address, 5) month, 6) year. Examples of this are: \cite{Zargham-2011}\cite{Sadler-2006}.}

\item \MR{When referring to \textbf{journal papers}, include the following information: 1) author names, 2) paper title, 3) full journal name, 4) volume, 5) number, 6) month, 7) pages, 8) year. Examples of this are: \cite{Shannon-1948}\cite{Boyd-2011}\cite{Zordan-2014}.}

\item \MR{For \textbf{books}, include the following information: 1) author names, 2) book title, 3) editor and edition, 4) year.}
\end{itemize}
%
\MR{Note that some of the above fields may not be shown when you compile the Latex file, but this depends on the bibliography settings (dictated by the specific Latex style that you load at the beginning of the document). You may decide to include additional pieces of information in a given bibliographic entry, but please, be consistent across all the entries, i.e., use the same fields. Exceptions are in the (rare) cases where some of the fields do not exist (e.g., the paper {\it number} or the {\it pages}).}

% !TEX root = template.tex

\section{Processing Pipeline}
\label{sec:processing_architecture}

We start off our analysis by preprocessing the collected signals within the MATLAB environment: we chose that framework because we find it is easier to operate with matrices. In this first step we import the data collected by sensors, which are given as .dat files, then we select the signals from on-body sensors and discard the others, so we replace the missing values by means of interpolation and, at last, we store them as .mat files.

Secondly, we import the preprocessed data in a \textit{Jupyter Notebook} and make the dataset suitable for the classification task: this consists for example of segmenting data into windows, scaling and normalizing raw signals.

Then, after the last step of preprocessing, we define and train a suitable learning model. This is respectively done for both the locomotion activity and gestures recognition, i.e. with two different sets of labels. This system, which is forced to learn also the null class together with the actual movements, is then compared to a different system where two models are deployed: in that case, the first one has the purpose of detecting activity while the second one classifies the type of movement, if detected. \MD{Aggiungere figura pipeline?}

\section{Signals and Features}
\label{sec:model}

The \textbf{OPPORTUNITY} dataset, succinctly introduced in section \ref{sec:related_work}, has been collected from four subjects accomplishing different Activities of Daily Life (ADLs). As highlighted before, both the subjects and the environment they moved in were meticulously monitored.
The process of acquisition consisted in 5 consecutive runs (named ADL1 to ADL5) that followed a predetermined script, plus a sixth run consisting of 20 repetitions of each of the distinct discrete activity present in the script. Then each vector of samples corresponding to a single time-step is labelled; in the following we'll refer to Task A when we consider an high-level modes of locomotion (\textit{Standing, Walking, Sitting, Lying}) while we'll refer to Task B2 for more specific arm gestures, 17 in total. To these tasks we need of course to add the \textit{Null Class} that we mentioned earlier in this paper: specifically, this label represents the state where the participant does nothing (or something that is not classified among the listed classes). 

The wireless sensors worn by the subjects (IMU - Inertial Measurement Unit) provided acceleration among the three-axes, rate of turn, magnetic field and orientation information; in addition, 12 accelerometers were placed on the subjects' parts of the body sensible to movements (arms, back, hips and feet). All these sensors for a total of 145 distinct acquired channels. \MD{Aggiungere immagine ometti con sensori?}

For the purposes of our work, however, we based the analysis only on on-body sensor signals using just a subset of the available sensors: in this way, we ended up with a total of 113 channels. Another important point is that in the preprocessing phase we performed spline interpolation (which uses a cubic polynomial) in channels that manifested missing data (equivalent to a NaN vale); however, this type of interpolation ends up meaningless if more than the 30\% of data is missing. For this reason we had to discard all the three columns corresponding to one of the physical devices: finally, this led us to work with 110 channels. Since we noticed that the head and tail of the measurement sessions correspond to a transient where most of the sensors are turned-off, we decided to discard them. In this way we ensure that the interpolation phase provides consistent results; the subsequent step consists in normalizing each column with respect to mean and variance. After that, in order to instruct the framework on one subject, we stack sessions from ADL1 to 3 and Drill as a first step to create our training set; then, we assemble also ADL4 and ADL5 to build the test set. 

To conclude the preprocessing phase, finally we decided to follow the same procedure as in several works we cited earlier: we apply in fact the \textit{sliding window} technique on the datasets, obtaining a tensor of windows constituted of \MD{...} samples (\MD{...} ms), using a stride of length \MD{...}. In our case, we decided to assign to each window the most frequent label: this doesn't constitute a problem per se, even when changing the size of the sliding window, as long as it is kept short enough for being representative of a movement.

\MD{ultime considerazioni su come una scelta adeguata di window size e stride possa influire notevolmente sulle performance, poi direi che questa parte è finita. Dovrei aggiungere qualcosa sulle features forse?}


\section{Learning Framework}
\label{sec:learning_framework}


One of the main problems in Human Activity Recognition is handling \text{inactivity}.

Thinking of a real recognition system, 
In this paper we compare two different learning strategies, mimicking a real system. In the first \ref{sub:oneshot}, \text{One Shot Classification}, the model is trained to learn a representation of the involved classes together with the null class

\subsection{One Shot Classification}
\label{sub:oneshot}

\subsection{Two Steps Classification}
\label{sub:twosteps}


% !TEX root = HDA_MDRL.tex

\section{Results}
\label{sec:results}

\begin{table}[]
	\begin{tabular}{lccccc}
		\textbf{Task A}				& S1	& S2	& S3	& [S2 S3]	& S4	\\
									&		&		&		&			&		\\
		F1-measure (with Null class)&		& 		&		&			& 		\\
		Baseline					& 0.85	& 0.86	& 0.83	& 0.85		& 0.77	\\
		Contributed					& -		& 0.85	& 0.81	& 0.83 		& - 	\\
		Our results					& 0.91	& 0.76	& 0.83	& 0.88		& 0.79	\\
									&		&		&		&			&		\\
		F1-measure (no Null class)	&		&		&		&			&		\\
		Baseline					& 0.86	& 0.86	& 0.85	& 0.85		& 0.76	\\
		Contributed					& -		& 0.90	& 0.87	& 0.87 		& - 	\\
		Our results					& 0.94	& 0.78	& 0.90	& 0.84		& 0.89	\\
	\end{tabular}
	\caption{Comparison of results on modes of locomotion (task A), with and without the \textit{Null} class. For each of the baseline, contributed and our models, here are reported only the best scores achieved.}
	\label{tab:res_A}
\end{table}
\begin{table}[]
	\begin{tabular}{lccccc}
		\textbf{Task B}				& S1	& S2	& S3	& [S2 S3]	& S4	\\
									&		&		&		&			&		\\
		F1-measure (with Null class)&		& 		&		&			& 		\\
		Baseline					& 0.85	& 0.89	& 0.86	& 0.87		& 0.88	\\
		Contributed					& -		& 0.88	& 0.87	& 0.88 		& 0.71 	\\
		Our results					& 0.89	& 0.81	& 0.88	& 0.84		& 0.85	\\
									&		&		&		&			&		\\
		F-measure (no Null class)	&		&		&		&			&		\\
		Baseline					& 0.55	& 0.53	& 0.58	& 0.56		& 0.48	\\
		Contributed					& -		& 0.72	& 0.80	& 0.77 		& 0.17 	\\
		Our results					& 0.81	& 0.40	& 0.82	& 0.64		& 0.70	\\
	\end{tabular}
	\caption{Comparison of results on gesture recognition (task B), with and without the \textit{Null} class. For each of the baseline, contributed and our models, here are reported only the best scores achieved.}
	\label{tab:res_B}
\end{table}

As in most of the works mentioned in section \ref{sec:related_work}, besides accuracy, we used $F_1$ measure to estimate the goodness of our models. Defining precision and recall as: 
\begin{equation}
	p = \frac{TP}{TP+FP} \qquad r = \frac{TP}{TP+FN}
\end{equation}
where \textit{TP = true positives}, \textit{FP = false positives} and \textit{FN = false negatives}, the $F_1$ measure is evaluated as the harmonic average between the two. In particular, since we deal with a multi-class problem we need to add a measure of weight to the $F_1$ equation:
\begin{equation}
	F_1 = \sum_i 2w_i \frac{p_i \cdot r_i}{p_i + r_i}
\end{equation} 
where the weights are defined as the number of samples of a particular class divided by the total number of samples $w_i = \frac{n_i}{N}$. The weighted measure can help also with the class imbalance problem; we must highlight however that our models are still trained on an imbalanced dataset, so in our opinion this could provide only a minor improvement.

\begin{figure}[t]
	\centering
	\includegraphics[scale=.4]{figure/A_models_nullclass}
	\caption{Task A : One-Shot Classification}
	\label{fig:A_os}
\end{figure}
\begin{figure}[t]
	\centering
	\includegraphics[scale=.4]{figure/A_models_cascade}
	\caption{Task A : Cascade Classification}
	\label{fig:A_casc}
\end{figure}

Since in \cite{Chavarriaga2013} results from the contributed methods have been compared to some baseline approaches, here we do the same by displaying only the best result for each type. In tables \ref{tab:res_A} and \ref{tab:res_B}
are reported respectively the results on task A and B, with and without the \textit{Null} class. Our results are taken then, for both tasks, from \textit{One-Shot Classification} and \textit{Activity Classification}, keeping only the best among the scores of the four models used.
We can see that sometimes our models perform better, particularly on \{S1, S3, S4\}, in other cases instead our score drops, as for \{S2\}, or is very similar to the others, as for \{[S2 S3]\}: it seems in fact that we have better results for those subjects for which contributed models in \cite{Chavarriaga2013} are worse, and viceversa. This is not though totally unexpected to us, since contributed models had to focus on \{S2, S3, [S2 S3]\}, while our attention has been more uniformly distributed.
What's surprising instead is what we obtained for S4, whose signals have been artificially affected by some rotational noise (task C in \cite{Chavarriaga2013}): we had good scores, compared to the others, but what's unexpected is that our performances for S4 are regularly better that the ones for S2.

In Figures \ref{fig:A_os} and \ref{fig:A_casc} we present the results regarding task A (modes of locomotion, high level movements) for each participant in the experiment: in those configurations we wanted to see if there was one architecture that evidently outperforms the others. As we can see, unfortunately, this is not the case since, among all the frameworks that we tested, the differences in terms of weighted $F_1$ measure are negligible: this suggests that the best model could be the simplest one, which is "Convolutional".
What we can clearly see instead are the variations among distinct subjects, since they are all investigated separately: $S1$ and $S4$ for example perform better in both the configurations, while $S2$ is typically the worst, as previously noticed.
The same considerations just discussed for task A are still valid for task B, as can be seen from the results showed in figures \ref{fig:B_os} and \ref{fig:B_casc}.

\begin{figure}[t]
	\centering
	\includegraphics[scale=.4]{figure/B_models_nullclass}
	\caption{Task B : One-Shot Classification}
	\label{fig:B_os}
\end{figure}
\begin{figure}[t]
	\centering
	\includegraphics[scale=.4]{figure/B_models_cascade}
	\caption{Task B : Cascade Classification}
	\label{fig:B_casc}
\end{figure}

Focusing instead on the differences between the two pipelines, i.e. One-Shot and Cascade classifications, we see that also in this case that performances are very similar: splitting the pipeline into a detection and a classification gives a slight improvement on both tasks, in the order of $1\%$. For ease of assessment we report the results in figures \ref{fig:A_comp} and \ref{fig:B_comp}.
\begin{figure}[t]
	\centering
	\includegraphics[scale=.6]{figure/A_pipeline_comparison}
	\caption{Task A : One-Shot and Cascade comparison}
	\label{fig:A_comp}
\end{figure}
\begin{figure}[t]
	\centering
	\includegraphics[scale=.6]{figure/B_pipeline_comparison}
	\caption{Task B : One-Shot and Cascade comparison}
	\label{fig:B_comp}
\end{figure}
Even though results for the Cascade classification don't seem to be that exciting, we can stay positive about them, keeping in mind that in our tests we always performed detection and classification with the same model, but there should be an increase in performances when using the best available model for each of the two tasks.



% !TEX root = HDA_MDRL.tex

\section{Concluding Remarks}
\label{sec:conclusions}

In our project we realized from scratch two different pipelines to perform activity recognition. A comparison between the two is carried out using different models, which are inspired to the best ones in literature. This helped us understanding that there isn’t a clear best choice from an accuracy point of view; in fact, in locomotion classification, results were slightly worse when the \textit{Null Class} was considered, while in gesture recognition it was the opposite, with a sensible decrease in accuracy when inactivity wasn’t considered. Despite our deep study, still one must examine case by case when to use one model or the other, with respect to the discussion that we developed in section \ref{sec:model}.

Future works could try to implement effectively the Two-Steps classification pipeline, by placing the classification model after the detection one: we tried ourselves, but results weren’t satisfactory enough to be reported here. A good approach to the problem could be trying to understand how the accuracies of the two single models add up when in cascade. In this paper in fact, when comparing the two pipelines, we didn’t keep into account the precision of the detection model in the Two-Step architecture; we considered instead only the classification performances without the null class to provide an estimate of the accuracies achieved using only deep learning models. 

Another thing that is left to be done, is to try to solve the class imbalance problem; a good proposal could be trying to replicate what has been done in \cite{japkowicz2002class}. In our work we only dealt with it for evaluation purposes, but something could be done also to improve the training phase.

This project was very helpful to us because we actually learned to code in python, we had to face different obstacles and learned that results are never what one would expect: we hoped in better results, for the advanced techniques used, and expected similar networks to have almost equal performances, which wasn't the case throughout our work.

%\bibliography{biblio}
%\bibliographystyle{ieeetr}

\end{document}


